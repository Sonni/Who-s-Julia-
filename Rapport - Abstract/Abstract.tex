\documentclass[a4paper,11pt]{article}

\usepackage[utf8]{inputenc}
\usepackage[T1]{fontenc}
\usepackage[danish]{babel}
\usepackage{mathptmx}

\usepackage{color}
\usepackage{float}
\usepackage{fancyvrb}

\usepackage{setspace}
\onehalfspacing

\title{First year project\\Who's Julia?\\\rule{10cm}{0.5mm}}
\author{Simon Lehmann Knudsen, simkn15\\Sonni Hedelund Jensen, sonje15\\Asbjørn Mansa Jensen, asjen15
\\ DM501\\\rule{5.5cm}{0.5mm}\\}
\date{\today}

\begin{document}

\maketitle

\vfill

\newpage
\section*{Abstract}
Julia programming language is a younger language which appeared in 2012. Julia resembles a lot like an older language, Python. The syntax is very close to Python, some code can almost be copy pasted from Python to Julia, and vice versa. But why make Julia, if it resembles a lot like Python? This is one of the interesting questions to be answered. The people behind Julia claims that it is much faster than Python, and in some cases on pair with some of the faster languages. The main focus of the project will be to implement different algorithms in Julia, and benchmark these with implementations in other languages. Benchmarking is to measure the performance of an object, in this case an algorithm. In order to measure the performance, the focus will be on measuring the running-time of algorithms, and compare the programming languages. The running-time can be divided into different categories, e.g. real time and CPU-time. Some programs has functions which occurs automatically in the background, to ease the user, but impacts the performance negatively. Study shows that experienced people are disagreed on how to properly benchmark a program in a certain language, and also as simple as which results to compare; median, best or worst time. Benchmarking properly can be quite difficult, since it seems like there's not one right answer on how to actually do it.

\end{document}